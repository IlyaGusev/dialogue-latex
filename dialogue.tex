\documentclass[a4paper]{article}

\usepackage{dialogue}

\date{}

\begin{document}

\settitle{russian}{Как написать статью}{Первый автор \and Второй автор}

%% \selectlanguage{russian}
%% \title{Как написать статью}
%% \author{Первый автор \and Второй автор}
\let\newpage\relax\maketitle

\selectlanguage{english}
\title{How to write a Research Paper}
\author{First Author \and Second Author}
\maketitle

\section{Examples}

I want this example to be useful.\footnote{But who can be sure?}

\subsection{Formul\ae}

What if I want a vector like $\vec{x}$?

\begin{equation}
  \text{similarity} = \cos(\theta) = {\mathbf{A} \cdot \mathbf{B} \over \|\mathbf{A}\| \|\mathbf{B}\|}
\end{equation}

\subsection{Table}

Table~\ref{tab:example} is a nice one.

\begin{table}[htbp]
\centering
\caption{\label{tab:example}Wow!}
\begin{tabular}{|c|c|c|}\hline
\LaTeX & \LaTeX & \LaTeX \\\hline
\LaTeX & \LaTeX & \LaTeX \\\hline
\LaTeX & \LaTeX & \LaTeX \\\hline
\end{tabular}
\end{table}

\subsection{Figure}

Figure~\ref{fig:example} makes your paper visually appealing.

\begin{figure}[htbp]
\centering
% The plot was obtained at https://people.sc.fsu.edu/~jburkardt/data/eps/eps.html.
\includegraphics[scale=.75]{mathematica.eps}
\caption{\label{fig:example}Great!}
\end{figure}

\subsection{Citations}

Citations are a good thing~\cite{RUSSE2018}.

\begin{thebibliography}{1}

\bibitem{RUSSE2018}
Panchenko A., Lopukhina A., Ustalov D., Lopukhin K., Arefyev N., Leontyev A., Loukachevitch N. (2018), RUSSE'2018: A Shared Task on Word Sense Induction for the Russian Language, Computational Linguistics and Intellectual Technologies: papers from the Annual conference ``Dialogue'', Moscow, Russia, pp.~547--564.

\end{thebibliography}

\end{document}
